\documentclass[12pt]{article}

\usepackage[russian]{babel}
\usepackage[utf8]{inputenc}

\title{Домашняя работа №1}
\author{Александра Кондратенко}
\date{}

\begin{document}
	\maketitle
	\begin{flushright}
		\textit{Audi multa,\\loquere pauca}
	\end{flushright}
	
	\vspace{20pt}
	Это мой первый документ в системе компьютерной вёрстки \LaTeX.
	
	\begin{center}
		\textsf{\huge{<<Уpa!!!>>}}
	\end{center}
	
	А теперь формулы. \textsc{Формула}~--- краткое и точное словесное выражение, определение или же ряд математических величин, выраженный условными знаками. 
	\vspace*{15pt}
	
	\hspace*{15pt}{\Large{\textbf {Термодинамика}}}
	
	Уравнение Менделеева--Клайперона~--- уравнение состояния идеального газа, имеющее вид $pV = \nu RT$, где $p$~--- давление, $V$~--- объём, занимаемый газом, $T$~--- температура газа, $\nu$~--- количество вещества газа, а $R$~--- универсальная газовая постоянная. 
	\vspace*{15pt}
	
	\hspace*{15pt}{\Large{\textbf {Геометрия \hfill Планиметрия}}}
	
	Для \textit{плоского} треугольника со сторонами $a$, $b$, $c$, и углом $\alpha$, лежащим против стороны $a$, справедливо соотношение
	$$
		a^2 = b^2 + c^2 - 2 b c \cos\alpha, 
	$$
	из которого можно выразить косинус угла треугольника:
	$$
		\cos\alpha = \frac{b^2 + c^2 - a^2}{2 b c}.
	$$
	
	Пусть $p$~--- полупериметр треугольника, тогда путём несложных преобразований можно получить, что 
	$$
		\mathrm{tg}\frac{\alpha}{2} = \sqrt{\frac{\left( p - b \right) \left( p - c \right)}{p \left( p - a \right)}}.
	$$
	
	\vspace{1cm}
	\begin{flushleft}
		На сегодня, пожалуй, хватит\dots Удачи!
	\end{flushleft}
\end{document}